\chapter{Introduction}

Ce document a pour objectif de guider l'utilisateur dans son installation de la plateforme GeOxygene sous Windows ou sous Linux. On décrira d'abord l'installation des outils nécessaires, puis dans un second temps nous préciserons les démarches d'import et de compilation du code source. 

\bigskip

Ce guide s'adresse principalement aux développeurs qui souhaitent faire leurs premiers pas dans GeOxygene.

\bigskip

Il est à noter que la procédure d'installation et de configuration expliquée ici se base sur la version 1.5 de GeOxygene.

%-----------------------------------------------------------
\section{Environnement}

GeOxygene est un projet Open Source écrit en JAVA. Le projet est généré à partir de Maven\footnote{http://maven.apache.org/}. 

\bigskip

L'environnement de développement utilisé est celui d'Eclipse\footnote{http://www.eclipse.org/}, éditeur très largement utilisé aujourd'hui pour les développements JAVA. Le plugin m2eclipse\footnote{http://eclipse.org/m2e/} fournit un support pour l'utilisation de Maven dans l'IDE.


%-----------------------------------------------------------
\section{Prerequis}

Pour pouvoir fonctionner, GeOxygene nécessite l'installation d'une version 6 (ou supérieure) d'une JDK. Il est possible de télécharger cet environnement sur le site de Sun à l'adresse suivante :

\medskip

\href{http://www.oracle.com/technetwork/java/javase/downloads/index.html}{http://www.oracle.com/technetwork/java/javase/downloads/index.html}
